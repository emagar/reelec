\noindent Here, $\vn{incumbent}_{it}$ is 1 if the sitting municipal president was on cycle $t$'s ballot, running for consecutive reelection, 0 otherwise; $\vn{party}_{it}$ is 1 if the candidate shares a party label with the outgoing municipal administration, 0 otherwise; $\vn{T}_{it}$ is a vector of observed time-varying covariates (such as the state of the economy as perceived by voters or the governor's party); and $\vn{X}_i$ is a vector of municipal time-invariant covariates (such as the mix of interest groups in the municipality or its level of education). Note that, with the exception of the constant $\alpha_{it}$ and the error term $\epsilon_{it}$, all regression coefficients are constant across three-year election cycles (the $t$ subindex is absent). Subtracting the first equation from the second yields

eq

\noindent Because time invariant covariates ($\vn{X}_i$) drop out, the danger of omited variable bias is much less than in cross-sectional estimation. Other covariates constant, coefficient $\iota$ translates a unit increase in $\Delta \vn{incumbent}_{i2}$ into vote share above or below last electoral cycle's. Expect $\iota$ votes above the baseline when an incumbent ran for reelection and the predecessor three years before did not. But expect the symmetric vote swing of $-\iota$ when the opposite holds: the predecessor ran for reelection while the current incumbent either retired, or was term-limited, or got no party's endorsement ($\Delta \vn{incumbent}_{i2}=-1$). So $\hat{\iota}$ (the estimated coefficient) estimates the personal (?) component of incumbency advantage, separate from the partisan.

Likewise, coefficient $\pi$ is the vote change effect of campaining with your party in the municipal presidency versus out ($\Delta \vn{party}_{i2}=1$). It also captures the effect of campaigning in the opposition versus in government ($\Delta \vn{party}_{i2}=-1$). The incumbency curse literature \citep{lucardi.rosas.Incumbency.2016,folkle.snyderGubMidtermSlump.2012} expects $\pi < 0$. More interesting is coefficient $\xi$ gauging interactive effects.  


Two-term limits impose the constraint that $\vn{incumbent}_{i1} \rightarrow \vn{incumbent}_{i2}=0$. The implication is that 
