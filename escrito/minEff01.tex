\section{La reforma}

Norma federal, variación subnacional


\section{La hipótesis de efectos mínimos}


\section{Modelo}

Un distrito $d$ tiene una distribución normal del voto partidista entre los partidos $a$ y $b$ y los independientes $i$:

$v_{a,d} + v_{b,d} + v_{i,d} = 1$ (ver compositional, hacer aleatorio centrado en media de regresión últimos años).

Un candidato que ha cultivado su voto personal atrae el voto independiente, de otro modo se distribuye según voto normal. (Quizás meter factor de corto plazo...)

El voto personal puede cultivarse en el crto plazo, o estar asociado con una maquinaria electoral ``privada.''

El partido tiene opción de impedir que el incumbent se renomine en el distrito. Si lo hiciera, el distrito se revertiría al voto normal. Donde el margen $m_d = v_{a,d} + v_{b,d} < v_{i,d}$, el partido corre el riesgo de perder el distrito. 

Reportar márgenes distritales en últimas elecciones federales.

Reportar volatilidad. 
